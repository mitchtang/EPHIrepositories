\documentclass[legno,11pt]{article}
%\usepackage[utf8x]{inputenc}
\usepackage[utf8x]{inputenc}
\usepackage{ucs}
\usepackage{amsmath}
\usepackage{amsfonts}
\usepackage{amssymb}
\usepackage{pifont}
%\usepackage{natbib}
\usepackage{fleqn}
%\usepackage{txfonts}
%\usepackage{hyperref}
%usepackage[numbers]{natbib}
%usepackage{elsarticle-harv}

%\usepackage{harvard}
\usepackage[left=3cm,top=2cm,right=3cm,nohead]{geometry}
\usepackage{amsmath}
\usepackage{amssymb}
\usepackage{amsthm}
\usepackage{amstext}
\usepackage{sectsty}
\allsectionsfont{\centering} \makeatletter
\def\@seccntformat#1{\csname the#1\endcsname.\hspace{0.2cm}}
\makeatother
\usepackage{amsfonts}
\usepackage[onehalfspacing]{setspace}

%\usepackage[sort&compress]{natbib}
%\bibpunct{}{}{,}{n}{}{;}

%$\usepackage[headings]{fullpage}
%\usepackage{setspace}

\doublespacing
%\usepackage[norule]{footmisc}
%
%\usepackage{hyperref}
%\usepackage{float}
%%%%%%%%%%%%%%%%%%%%%%%%%%%%%%%%%%%%%%%%%%%%%%%%%
\usepackage{longtable,lscape}
\usepackage{rotating,threeparttable}
\usepackage{dcolumn}
\usepackage{multirow}
\usepackage{booktabs}
\usepackage{dcolumn,booktabs}
\newcolumntype{d}[1]{D{,}{,}{#1}}
\newcolumntype{.}[1]{D{.}{.}{#1}}
\usepackage{graphicx}
\usepackage{caption}
\usepackage{amsfonts}
\usepackage{booktabs}
%\usepackage{graphicx,color,pstcol,pst-tree}
%\usepackage{pst-node,pst-tree}
%\makeatletter % Reference list option change
%\renewcommand\@biblabel[1]{} % from [1] to 1.
%\makeatother %


\renewcommand{\abstractname}{}
%%%%%%%%%%%%%%%%%%%%%%%%%%%%%%%%%%%%%%%%%%%%%%%%%%%%%%%%
\usepackage{graphicx}
\newcommand{\bpara}[4]{ % #1 x; #2 y; #3 angle; #4 height
\begin{picture}(0,0)%
\setlength{\unitlength}{1pt}%
\put(#1,#2){\rotatebox{#3}{\raisebox{0mm}[0mm][0mm]{%
\makebox[0mm]{$\left.\rule{0mm}{#4pt}\right\}$}}}}%
\end{picture}}

\abovecaptionskip=8pt \belowcaptionskip=0pt
\newtheorem{thm}{Theorem}[section]
\newtheorem{conj}[thm]{Conjecture}
\newtheorem{cor}[thm]{Corollary}
\newtheorem{lem}[thm]{Lemma}
\newtheorem{prop}[thm]{Proposition}
\newtheorem{exa}[thm]{Example}
\newtheorem{defi}[thm]{Definition}
\newtheorem{exe}[thm]{Exercise}
\newtheorem{rek}[thm]{Remark}
\newtheorem{assum}{Assumption}

%\linespread{1.9}
%\linespread{1.55}
%\input{tcilatex}
\begin{document}

\title{Household Preferences for Employer-Provided Health
Insurance: An Empirical Discrete Games Approach
\thanks{We thank Kamhon Kan, Wen-Jen Tsay, and the seminar participants at National Chung Cheng University, Academia Sinica, National Taipei University, the  $12^{th}$ Empirical Economics Conference at Taiwan, and the CEANA-NTU-TEA session at the 2012 Annual Meeting of American Economic Association for various comments and suggestions regarding this paper. Wan-Ting Chen, Wan-Ning Chiu and Tsong-Yuen Hu provided research assistance. The usual disclaimer applies.}
}
\author{Ji-Liang Shiu
\thanks{%
Hanqing Advanced Institute of Economics and Finance, Renmin
University of China, Beijing 100872, P.R.China. Email:
jishiu@ruc.edu.cn.}and Meng-Chi Tang\thanks{Department of Economics,
National Chung-Cheng University, 168 University Rd. Min-Hsiung
Chia-Yi, Taiwan. Email: ecdmct@ccu.edu.tw. } }

\date{\today}
\maketitle

\begin{abstract}
This paper examines how a spouse's access to employer-provided
health insurance (EPHI) affects preferences to acquire the same
access at both the individual and household level. Regarding
household joint EPHI access decisions in the labor market
equilibrium as a cooperative, simultaneous game between husband and
wife, this paper shows how this joint decision is affected by the
consideration of family bargaining power balance. This study
investigates this joint decision using panel data from the
Medical Expenditure Panel Survey. Our empirical estimation
controls for unobserved fixed effects, and allows the EPHI
access of both husband and wife to be endogenous.
Empirical results show that spousal access to EPHI is
negatively related to own EPHI access, indicating that
household members as a unit prefer single EPHI access
to dual access. Using family income share as a proxy for
individual indirect utility, this study shows that the husband's incremental disutility of having
spousal EPHI access is greater than that of the wife.
This study explains the results by considering the bargaining power balance within households, which provides a new perspective
on how married couples make joint EPHI access decisions.

%This paper examines how a spouse's access to employer-provided
%health insurance (EPHI) affects preferences to acquire the same
%access at both the individual and household level. We consider this
%strategic interaction as power balance within household under a
%cooperative game framework, and allow the spouse's EPHI access to be
%endogenous using panel data models. Empirical results show that
%spousal access to EPHI is negatively related to own EPHI access, and
%the husband's incremental disutility of having spousal EPHI access
%is greater than that of the wife. This paper provides a new
%perspective on household EPHI joint decisions.
\end{abstract}

%\textbf{Keywords:} employer-provided health insurance; family bargaining power; joint decision; gender differences
%\thispagestyle{empty}

%\pagestyle{plain}

\newpage
\section{Introduction}\label{section1}
test man test
does tthis work mannnnn

While marital status is an important factor in individual health
status and employment decisions, previous studies on these decisions
focus almost exclusively on men. This is primarily because the labor
force participation rate of women used to be low, and a fairly large
percentage of women were insured through their husbands. Therefore,
the potential behavioral effects of women in this type of
decision-making were relatively insignificant. Nevertheless, due to
recent dramatic increase in female labor force participation, many
individuals now live in households where both husband and wife work.
As a result, many families have two potential sources of health
insurance coverage from both husband's and wife's employers, and the
number of households in which women are the primary insurers of
their families is also increasing. Consequently, it is important to
study husband's and wife's health insurance and employment decisions
jointly at the household level.\footnote{Abraham and Royalty (2005)
emphasized the importance of considering this type of joint
decisions at the household level. Their arguments are supported by
empirical evidence showing that having a second earner in household
significantly changes the household health insurance options. }
\par
This study focuses on household joint decisions about
employer-provided health insurance (EPHI). EPHI is popular in the
United States because it is usually much more generous than the
health insurance provided in the non-group market.\footnote{In the
United States, approximately 71\% of non-elderly Americans are
covered by private health insurance. Of that total, 90\% are covered
through EPHI.} Employers can also reduce adverse selection problems
and administrative expenses by pooling healthy and unhealthy
employees.\footnote{The costs of health insurance in large firms is
almost 35\% lower than in small firms.} In addition, there is a tax
incentive for employers to provide health insurance.\footnote{See
Woodbury and Huang (1991), Gruber and Poterba (1996), Gentry and
Peress (1994), and Royalty (2000) for related discussions.} However,
it is not clear why some families have two earners with dual EPHI
access, since most of EPHI provides family coverage. Royalty and
Abraham (2006) argued that there may be some tradeoffs to having a
job with EPHI, such as more hours of work. For example, if one
member has a job with EPHI access, the other member must take over
more of the housework or child caring activities, and is therefore
less likely to have a job with EPHI access. In this paper, having
the access to EPHI can be considered as a source of family
bargaining power.\footnote{Having EPHI access can be incorporated
into McElroy (1990)'s framework as an extrahousehold environmental
parameter in the bargaining power function.} One may argue that the
EPHI decision might have much less to do with the bargaining power
within families. The EPHI decision is just a result of preferences
for personal work. This view ignores that the potential health and
monetary benefits come from EPHI and family members of EPHI holders
receive higher benefits or has greater access to resources.
Providing EPHI to families often carries more influence in family
decision-making and the extent to which the EPHI decision is related
to power shift within a family.

Since individual bargaining power within the household represents
personal well-being at the threat point in the cooperative Nash
bargaining model, access to EPHI is an important source of family
bargaining power. This creates an incentive for family members to
acquire a job with EPHI access regardless of whether spousal EPHI
access is available. Therefore, how husband and wife make their
health insurance joint decision
when facing these potential tradeoffs is ambiguous.%
\footnote{Monheit et al. (1999), Marquis and Kapur (2004), and
Abraham et al. (2005) provided explanations why two-earners family
may choose double coverage by looking at the actual EPHI coverage
rather than access to it. There may be some cost advantages for a
family to have dual EPHI coverage, and different EPHI programs may
also provide different coverage plans that family needs. Abraham et
al. (2002) also investigated the differences in behavior between
household with dual and single EPHI access when estimating a model
of household demand for health insurance. Dey and Flinn (2008)
showed there is an option value to find a job with EPHI access even
when spousal EPHI access is offered, because it is likely that
spouse may lose his/her job in the future. Following Royalty and
Abraham (2006), this study focuses on whether individual has EPHI
access, but not whether the individual holds the EPHI offer.}
%\textbf{Shiu Oct03
%2012}\footnote{One may argue that the EPHI decision might have much
%less to do with the bargaining power within families. The EPHI
%decision is just a result of preferences for personal work. This
%view ignore that the potential better health and monetary benefits
%brought from EPHI and which family member receives a higher benefits
%or has greater access to resources. Providing EPHI to families often
%carries more influence in family decision-making.}
\par
The principal interest of this paper is to investigate how husband
and wife behave strategically regarding their EPHI access decisions.
While other beneficial job characteristics such as sick pay or
pension plan may also affect their labor force participation
decisions, this study controls the impact from these determinants
using various econometric methods.  Besides, this study assumes that
EPHI is always offered with jobs, and players either accept or
reject this offer. This assumption is based on the observation that,
while an EPHI offer is contingent on the individual job search
outcomes, empirical analysis estimates an average effect of personal
EPHI access on spousal choice across every possible outcome among
individuals and time periods.\footnote{The modelling strategy
assumes that jobs with EPHI are always supplied and, hence whether
someone takes up such a job is simply a matter of personal and/or
family preferences.} Players in this framework therefore act like
representative agents who possess the freedom to choose between
accept or reject the EPHI offer. This static setting stresses the
strategic interactions between husband and wife regarding their EPHI
access, but not the dynamic aspects of their job searching behavior.
On the contrary, Dey and Flinn (2005, 2008) focused on the effect of
EPHI access on husband's and wife's job search and wage bargaining
behaviors, where individual job characteristics and health insurance
status are jointly determined as functions of the underlying labor
market environment.
\par
This study models this research question using the cooperative-game framework proposed by Bresnahan and Reiss (1991, henceafter BR). In their paper, BR shows how a simple cooperative game structure provides the economic rationale behind the dummy endogenous variable model in econometrics. Specifically, BR models husband's and wife's labor force participation decisions as a cooperative game played within household, because this discrete participation decision of each spouse affects the decision of the other spouse. Naturally, various decisions within household are made jointly by husband and wife, including their choice to find a job with EPHI access. BR's model not only provides the framework to analyze this type of household joint decisions, the cooperative game structure also stresses the joint nature of this type of decisions.
\par

This is the first study to apply BR's cooperative-game framework to
investigate household joint decisions.\footnote{Most of the related studies apply BR's
simultaneous-game framework to study firm's market entry decision. For example, see the survey by Reiss and Wolak (2007).} Specifically, this study applies BR's approach by
modeling the household health insurance joint decision process as a cooperative game between husband and wife. This
empirical
 framework assumes that husband and wife choose their
 optimal strategy of EPHI access based on the
 household joint utility generated by their actions,
 and whether an individual acquires personal EPHI access
  depends on their own, and the spouses' utilities of
  being the health insurance provider for the family. Under
  these settings, the optimal EPHI access within a
  household generates the highest joint utility for the family.
  This study estimates the
  threshold conditions for families to choose among none,
  single, or dual EPHI access within household, providing an explanation
  as to why families choose different
  numbers of EPHI access.
\par

Researchers have studied the effects of spouse's EPHI access on
labor market outcomes from various perspectives. For example, some
studies examine the relationship between husbands' health insurance
and their spouses' hours of work (Buchmueller and Valletta, 1998;
Olson, 1998; Wellington and Cobb-Clark, 2000; Schone and Vistnes,
2001; Bhargavan, 2000). While these studies find that the husband's
health insurance from work has significant negative effects on the
wife's labor involvement, most of these studies do not consider the
joint decision making at the household level, and assume the
husband's health insurance is exogenous to the wife's labor force
participation decision.\footnote{See Currie and Madrian (1999) and
Gruber (2000) for a comprehensive discussion of these studies.}
Other studies consider the spouse's health insurance access as
endogenous and instrument this access based on either spouse's job
characteristics (Schone and Vistnes, 2001), or spouse's human
capital characteristics (Olson, 2000; Honig and Dushi, 2005; Royalty
and Abraham, 2006). These job or human capital characteristics might
be appropriate instruments only for the unobserved heterogeneity in
cross section data. However, they may not provide enough variation
to capture time-varying endogenous variables in panel data models,
such as the spousal EPHI access. Weak instrument variables (IV) may
produce a bias on the estimates that only include time invariant
variables as IVs in panel data models.\footnote{The connection
between these unobserved factors and an individual's job or human
capital characteristics is readily apparent. Royalty and Abraham
(2006) noted that since many labor market outcomes depend on
household income, it is important to include reported household or
individual income to capture income effects on these labor market
outcomes. If these measures of income are misreported or imperfectly
measured, it will be difficult to obtain valid measures of these
quantities. This measurement error problem also generate bias in
instrument variable (IV) estimates that use spouse's job or human
capital characteristics as instruments for spouse's EPHI access.}
\par

To address the above econometric concerns, this study applies an
empirical framework to the data from Medical Expenditure Panel
Survey (MEPS) Panel 10 household component. The panel structure of
this data makes it possible to control for time invariant unobserved
effects within household, eliminating the permanent unobserved
effects. This econometric model also allows the possibility that
spousal EPHI access is endogenous to personal EPHI access.  The IV
method is then applied to reduce the endogeneity bias induced by
transitory unobserved effects, where two types of instruments are
used: spouse's human capital characteristics, such as age and
education; and spouse's firm characteristics, such as industry group
and firm size.
\par

Empirical results indicate that spousal EPHI access is negatively
related to personal EPHI access. An individual is five
 to six percent less likely to have EPHI access if
 his/her spouse already has EPHI. These estimates imply that
  husband's and wife's EPHI access does not complement each
   other due to their bargaining power conflicts, and
households on average prefer single EPHI access to dual access.
Since the household utility generated from dual EPHI access is equal
to the sum of individual gains from EPHI access plus the conflicting
loss to the family, the estimates suggest that the conflicting loss
of dual EPHI access is bigger than the gains. This implication is
further supported by empirical results using household income share
as a proxy of individual indirect utility. In particular, estimates
show that, while having spouse's EPHI access reduces both husband
and wife's preferences to having own EPHI access, the husband's
disutility from having spousal EPHI access is greater than that of
the wife. These estimates suggest a more traditional spousal
relationship
 is prevalent
where husbands have more bargaining power in the family.
Specifically, while the husband's access to EPHI does not greatly
change the original power balance within household, the wive's
access to EPHI significantly alters the balance. This increases the
incentive for husband to choose a job with EPHI offer to maintain
his position in the family. Therefore, all else being equal, having
the husband as the sole EPHI provider for the family brings the
maximum joint utility to households among all possible scenarios.
\par

The following section illustrates household EPHI decisions using a
simultaneous, cooperative game model. Section \ref{section3}
describes the MEPS data analyzed in this study, while Section
\ref{section4} proposes the empirical models estimating the effects
of EPHI access. Section \ref{section5} presents the empirical
results, and Section \ref{section6} concludes the paper.
\par

\section{Conceptual Framework}\label{section2}

This section constructs a cooperative game framework for empirical
estimation following BR's paper. Players are considered to cooperate
with each other by maximizing their joint utility under perfect
information assumption. Players play the game once and
simultaneously make their decisions. This setting is suitable for
current study because of our data characteristics. Specifically,
players do not play the game repeatedly due to the data's short
span. The data also provides no information on the sequence of
choices made by husband and wife. While the analysis would be biased
by imposing wrong assumptions on the sequence of decisions,
simultaneous-move assumption is appropriate because of our focus on
average effects. Similarly, the strategies of husband and wife are
assumed to include either to accept or to reject their EPHI offer.
As mentioned in Section \ref{section1}, while this choice is
contingent on the job search outcome, empirical analysis allows the
framework to simplify this choice by controlling the effects from
other determinants, and estimating the average effect across
individuals and time periods. Consequently, players in this
framework are considered as accept the EPHI offer if he/she is
observed to have an EPHI offer in the data; and individuals who are
either unemployed or employed without EPHI offer are treated as
reject the offer.
\par
Denote the utility functions of husband and wife by $\text{U}^{h}$
and $\text{U}^{w}$, respectively, and let $a^{k}=0$ indicate the
person chose to reject EPHI access, and $a^{k}=1$ otherwise, where
$k=h,w$. Since each member's utility in the household depends on
joint EPHI decisions, this study assumes that
$\text{U}^{h}=\text{U}^{h}(a^{i},a^{j},Z)$ and
$\text{U}^{w}=\text{U}^{w}(a^{i},a^{j},Z)$, where $Z$ represents
some characteristics of the household members and market
outcomes.\footnote{The $Z$ describes all attributes in the labor
market equilibrium other than household EPHI access decisions.}
Household members then choose the optimal solution, $a^{h}$ and
$a^{w}$, by maximizing the sum of their joint utility functions,
$W(a^{i},a^{j},Z)$=$\text{U}^{h}$+$\text{U}^{w}$, subject to the
household budget constraint.\footnote{This model can be considered
as a special model for family bargaining games (Kooreman and
Wunderink, 1997):
\begin{align}
&\max\limits_{a^{i},a^{j}} W(U^{h}(a^{i},a^{j},Z),U^{w}(a^{i},a^{j},Z))\label{family_utility}\\[-8pt]
&\text{s.t.}U^{h}(a^{i},a^{j},Z)\geq T^{h}(a^{i},Z)\notag\\[-8pt]
&\hspace{1mm}\quad U^{w}(a^{i},a^{j},Z)\geq T^{w}(a^{j},Z).\notag
\end{align}
where $T^{k}$ is the threat point of the agent obtained in the case
of a noncooperative Nash equilibrium. The Nash bargaining model uses
the household utility function, $N \equiv (U^{h}(a_{h},a_{w},Z)-
T^{h}(a_{h},Z))(U^{w}(a_{w},a_{h},Z)-T^{w}(a_{w},Z))$, but this
study adopts an additive separate utility function for estimation purpose.}
Similar to BR's approach, this study solves husband's and wife's
choices as reduced-form functions of $Z$, depending upon whether the
household members have no EPHI access, single access, or dual
access. Substituting those choice functions in the individual utility
functions yields the following indirect utility functions for
husband and wife:
\begin{align}
\text{U}^{h}&= \overline{\text{U}}^{h}+\text{a}^{h}\Delta_{1}^{h}+\text{a}^{w}\Delta_{2}^{h}+\text{a}^{h}\text{a}^{w}\Delta_{3}^{h}\label{indirect_1}\\
\text{U}^{w}&=
\overline{\text{U}}^{w}+\text{a}^{w}\Delta_{1}^{w}+\text{a}^{h}\Delta_{2}^{w}+\text{a}^{w}\text{a}^{h}\Delta_{3}^{w}.\label{indirect_2}
\end{align}

In the equations above, $\overline{\text{U}}^{k}$ denotes the
individual $k$'s utility when neither husband nor wife has EPHI
access. It captures the proportion of their indirect utilities that are measured by observable personal characteristics such as age and education, which represents the individual threat point in a family bargaining game mentioned above. $\Delta_{1}^{k}$ denotes the change of individual $k$'s
utility when $k$ becomes the sole EPHI provider for the family. It measures how much the individual $k$'s threat point is lifted with EPHI access, and therefore his/her bargaining power at home. $\Delta_{2}^{k}$, on the contrary, measures how spousal EPHI access affects personal indirect utility thus bargaining power at home. Spousal EPHI access reduces personal indirect utility and bargaining power because the family member who provides the health insurance coverage for the family has a greater say.\footnote{For example, spousal EPHI access affects the division of family chores. Since jobs with EPHI offer are often more demanding, $\Delta_{2}^{k}$ measures the individual marginal (dis)utility from sharing more family chores.} $\Delta_{3}^{k}$ measures the individual utility change when
both husband and wife have EPHI access. Since it is impossible to raise the bargaining power of husband and wife simultaneously, there are some conflicts of power in a family with dual EPHI access.\footnote{Dey and Flinn (2008) argued that there is an option value of personal EPHI access even when spousal EPHI access is offered. For example, there is a risk that spouse may lose his/her job in the future. The framework proposed here does not capture that effect due to the static setting. However, this framework allows us to estimate the strategic interactions between husband and wife.}
Therefore, the signs and magnitudes of the $\Delta^{k}$'s
identify the effects of personal and spousal EPHI access on
individual $k$'s preferences to find a job with EPHI access. It is expected that $\Delta_{1}^{k}$ is positive while $\Delta_{2}^{k}$ and $\Delta_{3}^{k}$ are negative.
\par
Given a vector of household characteristics and market outcomes $Z$,
there are four total payoffs of husband and wife in this cooperative
game, $W(a^{h},a^{w})$, conditional on own and spouse's EPHI access:
\begin{eqnarray}
W(0,0)& = &\overline{\text{U}}^{h}+\overline{\text{U}}^{w}  \\
W(1,0)& = &\overline{\text{U}}^{h}+\overline{\text{U}}^{w}+\Delta^{h}_{1}+\Delta^{w}_{2} \\
W(0,1)&  = &\overline{\text{U}}^{h}+\overline{\text{U}}^{w}+\Delta^{w}_{1}+\Delta^{h}_{2} \\
W(1,1)&  = &\overline{\text{U}}^{h}+\overline{\text{U}}^{w}+\Delta^{h}_{1}+\Delta^{h}_{2}+\Delta^{h}_{3}+\Delta^{w}_{1}+\Delta^{w}_{2}+\Delta^{w}_{3}.
\end{eqnarray}
\par

Individual then chooses between rejecting or accepting EPHI offer by
maximizing total payoff of the family. For example, conditional on
wife's EPHI access, husband chooses a job with EPHI when $W(1,0)$ is
greater than
 $W(0,0)$, and when $W(1,1)$ is greater than $W(0,1)$. Consequently, husband's and wife's decision rule can be summarized as follows:
\begin{align}
\text{a}^{h}&= 1[\Delta_{1}^{h}+\Delta_{2}^{w}+\text{a}^{w}(\Delta_{3}^{h}+\Delta_{3}^{w})\geq 0]\label{oya_1}\\
\text{a}^{w}&= 1[\Delta_{1}^{w}+\Delta_{2}^{h}+\text{a}^{h}(\Delta_{3}^{h}+\Delta_{3}^{w})\geq 0]\label{oya_2}.
\end{align}
\par

 Following BR's approach,
this study treats $\Delta_{3}^{k}$+$\Delta_{3}^{l}$ as constants
  (i.e., invariant across households) and
   $\Delta_{1}^{k}+\Delta_{2}^{l}$ as random variables
   that vary across households, where $k\neq l$.
Assume that
   $\Delta_{1}^{h}+\Delta_{2}^{w}$ is a linear function
   of observables and individual heterogeneity,
   $\Delta_{1}^{h}+\Delta_{2}^{w}=X^{k}\beta^{k}+\eta^{k}_{i}-u^{k}$, which implies that the incremental utility for
   household with only one EPHI access can be explained by the holder $k$'s observable characteristics and
   some unobservables that vary between different households.

   The
        structural equations determining the equilibria
        of this game therefore can be
        formulated as follows:
\begin{align}
\text{a}^{h}&= 1[X^{h}\beta^{h}+\text{a}^{w}(\Delta_{3}^{h}+\Delta_{3}^{w})+\eta_{i}-v^{h}\geq 0]\label{offer_1}\\
\text{a}^{w}&=
1[X^{w}\beta^{w}+\text{a}^{h}(\Delta_{3}^{h}+\Delta_{3}^{w})+\eta_{i}-v^{w}\geq
0].\label{offer_2}
\end{align}
\par
As mentioned in BR's paper, this framework identifies only
 the household joint preferences for EPHI access,
 $\Delta_{3}^{k}$+$\Delta_{3}^{l}$. Individual preferences
  for spouse's EPHI access, such as $\Delta_{2}^{k}$ and
  $\Delta_{3}^{k}$, cannot be separately determined under this framework.
  This is because husband's and wife's indirect utilities
  in this setting are unobserved and function as latent variables behind the binary choices to be estimated.
While $X^{h}$ and $X^{w}$ are the vectors of observable variables
connected to household characteristics and market outcomes,
$\eta_{i}$ is a time-invariant unobservable variable which may be
correlated with $X^{h}$ and $X^{w}$. If $X^{h}$ and $X^{w}$ include
individual's employment status, job tenure, wage, fringe benefits
such as pension plan, age, education, race and health status, there
may exist some tradeoff or correlation between these variables and
EPHI access. The inclusion of the unobserved heterogeneity
$\eta_{i}$ may break the possible correlation between $X^{h}$ and
$X^{w}$ and the unobserved error term. We argue that after
controlling for the unobserved individual heterogeneity, $X^{h}$ and
$X^{w}$ become exogenous. The econometric models section below
applies the household income share as a proxy for indirect utility,
making it possible to directly estimate the individual
   preferences for personal and spousal EPHI access.

\section{Data}\label{section3}

This study uses data from the Medical Expenditure Panel Survey
(MEPS) Panel 10 household component to test the proposed model. The
MEPS-HC collects detailed information from each household, including
individual demographic characteristics, health conditions, use of
medical services, health insurance coverage, income, and employment
characteristics. Each household was observed in five rounds of
interviews each year for two years (2005-2006), which makes it
possible to investigate household joint decisions using the panel
data methods.
\par

The household sampling criteria adopted in this study is a married
couple in which both husband and wife are aged between 20 and 64,
and at least one of them is employed during the survey period.
Individuals in the armed forces, disabled, or self-employed were
excluded from the sample. A total of 1384 observations satisfied
these criteria. This study classifies the households into four
groups based on EPHI status: households in which both husband and
wife have EPHI access; households in which only the husband has
access; households in which only the wife has access; and households
with no EPHI access. Table 1 defines the variables used in this
paper, and Table 2 presents the descriptive statistics for each
group based on their personal income, human capital, job, family
characteristics, and geographic distributions.
\par

\begin{center}
[Table 1 and 2 inserts here.]
\end{center}

The summary statistics provide some preliminary information about
family members' characteristics and EPHI access. While households
without EPHI access have the lowest average personal income among
all groups, husbands or wives with the highest income levels among
all groups are the sole providers of EPHI for their families. This
suggests a positive relationship between personal income and EPHI
access. In addition, while the difference between wife's average
income share from families with dual EPHI access versus families
with only her EPHI access is -.106 (.476-.582), the difference
between husband's average income share from families with dual EPHI
access versus families with only the husband's EPHI access is -.232
(.524-.756). These statistics may suggest that husbands generally
have stronger preferences for being the sole EPHI provider than
wives. However, it is also likely that husband's higher educational
levels, also shown in Table 2, probably point to better labor market
opportunities.\footnote{The same observation may also be generated
by a framework that allows for a labor market search environment
where men have a lower exogenous layoff rates than women.}
Therefore, more empirical works are required to study these
possible explanations.
\par

Table 2 summarizes the human capital characteristics within each
group.
In addition, husbands and wives in households with dual EPHI access have
the highest average education level among four groups, while
husbands and wives in households with only the wife's EPHI access
are the oldest on average. These observations suggest that age and
education may be correlated with family health insurance decisions.
This study also sets a health indicator that equals one if an
individual reports his/her health status as very good or
excellent.\footnote{The perceived health status variable in the MEPS
have five categories: excellent, very good, good, fair, and poor.}
Households with dual EPHI access have the highest percentage of
healthy couples, while households without EPHI access have the
lowest percentage of healthy couples. These statistics suggest a
positive relationship between family's health insurance decisions
and household members' health status.
\par

Table 2 also reveals the job characteristics of household members in
each group. In households with only one spouse's EPHI access, about
70 per cent of the husbands are employed while only 47 per cent of
the wives are employed. These statistics suggest that there may be
significant differences between husband's and wife's labor supply
decisions when their spouses possess the EPHI access. In addition,
couples in households with dual EPHI access have the longest job
tenure among all groups on average. Husbands and wives have the
highest hourly wage in households where they are the sole EPHI
provider, with more fringe benefits including pension plan and sick
pay. These statistics imply that EPHI access may serve as a good job
characteristic for husbands and wives.\footnote{In our estimation,
job characteristics such as tenure, wage, pension plan, and sick pay
are included to control the association between the EPHI offer and
"good jobs". Our estimation reflects a different pathway of joint
EPHI decisions other than "good jobs".}
\par

Finally, Table 2 also indicates that households with dual EPHI access
have the fewest number of children among all groups, while
households with no access to EPHI have the most children. Households
with only the husband's EPHI access also have more children than
households with only the wife's EPHI access.
These statistics provide a primitive evidence supporting the
negative relationship between spousal EPHI access and the amount of
housework and child caring activities.
These statistics also contradict the wives' employment rates in the sample,
which suggest that women may give up employment to take care of
children and re-enter employment after their children grow up. This
life-cycle interpretation is plausible because wives are older in
the groups with only the wife's EPHI access.
\par

\section{Econometric Models}\label{section4}
This section proposes an econometric model to estimate the household
preferences for the EPHI access, based on the cooperative game
framework discussed in the Section 2. The proposed econometric model
first estimates how spouse's access to EPHI affects own access to
EPHI in Eq. (\ref{offer_1}) \& (\ref{offer_2}).  Since the EPHI
access decision is a binary choice, a bivariate probit model is an
appropriate approach for estimating Eq. (\ref{offer_1}) \&
(\ref{offer_2}). To estimate the individual preference for household
EPHI access, the proposed econometric model uses household income
share as a proxy for husband's and wife's indirect utility in Eq.
(\ref{indirect_1}) \& (\ref{indirect_2}). This model considers the
potential endogenous problem between personal and spousal EPHI
access and income share because EPHI access may be correlated with a
variety of individual and job characteristics. This econometric
model takes advantage of the panel data structure of the MEPS by
allowing unobserved, time-invariant heterogeneity within households.

\subsection{Household Preferences for Employer-Provided Health Insurance Access}\label{section4_1}

The key issue in estimating the effects of spousal EPHI access on
personal access in Eq. (\ref{offer_1}) and (\ref{offer_2}) is to
allow the access of both husband and wife to be endogenous. The
outcome of EPHI access is a binary response that reflects whether an
individual is seeking a job with this fringe benefit. This study
uses a bivariate probit model to account for unobserved
heterogeneity and endogeneity under the panel data
setting.\footnote{A detailed discussion of the models appears in
Sections 15.7.3 and 15.8.5 in Wooldridge (2010).} The proposed
econometric model for Eq. (\ref{offer_1}) and (\ref{offer_2}) is
\begin{align}
\text{EPHI}_{it}&= 1[\textbf{x}_{it1}'\delta_{1}+\delta_{S}\text{SEPHI}_{it}+\eta_{i1}+ v_{it1}>0],\label{biprobit_11}\\ %\qquad
\text{SEPHI}_{it}&= 1[\textbf{x}_{it}'\delta_{2}+\eta_{i2}+ v_{it2}>0],\label{biprobit_12} %\qquad
\end{align}
with
\begin{align}
v_{it1}&\Big|\textbf{x}_{it1},\text{SEPHI}_{it},\eta_{i1}\sim\text{Normal(0,1)},\\
v_{it2}&\Big|\textbf{x}_{it},\eta_{i2}\sim\text{Normal(0,1)}.
\end{align}
In the model above , $1[\cdot]$ is an indicator function;
$\textbf{x}_{it}=(\textbf{x}_{it1},\textbf{x}_{it2})$, where
$\textbf{x}_{it1}$ is a vector of control variables that may affect
health insurance offer, and $\textbf{x}_{it2}$ is a set of exogenous
variables that relate to personal but not spousal EPHI access; and
$\eta_{i}=(\eta_{i1},\eta_{i2})$ captures the effects of unobserved
factors or omitted time-constant variable. Both $\text{EPHI}_{it}$
and $\text{SEPHI}_{it}$ are binary outcomes that indicate whether an
individual $i$ and his/her spouse has EPHI access, respectively.
Moreover, holding other factors fixed, $\delta_{S}$ in Eq.
(\ref{biprobit_11}) measures $\Delta_{3}^{h}+\Delta_{3}^{w}$ in Eq.
(\ref{offer_1}) and (\ref{offer_2}), which captures the effects of
spousal EPHI access on personal EPHI access. Since
$\text{SEPHI}_{it}$ is a potential endogenous variable, this study
instruments the variable by including $\textbf{x}_{it2}$ in Eq.
(\ref{biprobit_12}). While parameter $\delta_{S}$ in this bivariate
probit model only provides information about whether
$\text{SEPHI}_{it}$ has a positive or negative effect on
$\text{EPHI}_{it}$, this study estimates the average partial effect
(APE) to capture the magnitude of this effect.
\par

Since there may be possible correlation between $\textbf{x}_{it1}$
and $\eta_{i}$, this study adopts the correlated random effect
approach in Chamberlain (1982, 1984) and replaces $\eta_{i1}$ and
$\eta_{i2}$ with their linear projections onto the explanatory
variables in all time periods. Let
$\textbf{x}_{i}=(\textbf{x}_{i1},...,\textbf{x}_{iT})'$ , where
$\bar{\textbf{x}}_{i}$ contains the time averages of all strictly
exogenous variables. The correlated random effect modeling of
$\eta_{i}$ is shown as follows:
\begin{align}
\eta_{i1}|\textbf{x}_{i}&=\psi_{10}+\psi_{11}\bar{x}_{i1}
+\psi_{12}\bar{x}_{i2}+ ...+\psi_{1K}\bar{x}_{iK}+a_{i1}\equiv\bar{\textbf{x}}_{i}'\psi_{1}+a_{i1},\\
\eta_{i2}|\textbf{x}_{i}&=\psi_{20}+\psi_{21}\bar{x}_{i1}
+\psi_{22}\bar{x}_{i2}+
...+\psi_{2K}\bar{x}_{iK}+a_{i2}\equiv\bar{\textbf{x}}_{i}'\psi_{2}+a_{i2}.
\end{align}
Plugging these equations into Eq. (\ref{biprobit_11}) and
(\ref{biprobit_12}) leads to the following equations,
\begin{align}
\text{EPHI}_{it}&= 1[\textbf{x}_{it1}'\delta_{1}+\delta_{S}\text{SEPHI}_{it}+\bar{\textbf{x}}_{i}'\psi_{1}+a_{i1}+ v_{it1}>0],\label{biprobit_13}\\ %\qquad
\text{SEPHI}_{it}&= 1[\textbf{x}_{it}'\delta_{2}+\bar{\textbf{x}}_{i}'\psi_{2}+a_{i2}+ v_{it2}>0].\label{biprobit_14} %\qquad
\end{align}
To apply the bivariate probit model, standardize $a_{i1}+ v_{it1}$
and $a_{i2}+ v_{it2}$ as a standard normal distribution, and set
$e_{itj}=(a_{ij}+ v_{itj})/(1+\sigma^{2}_{a_{ij}})^{1/2}$ for
$j=1,2$, where each $e_{itj}$ has a standard normal distribution
conditional on $\textbf{x}_{i}$. Equations (\ref{biprobit_13}) and
(\ref{biprobit_14}) therefore become
\begin{align}
\text{EPHI}_{it}&= 1[\textbf{x}_{it1}'\delta_{1a}+\delta_{Sa}\text{SEPHI}_{it}+\bar{\textbf{x}}_{i}'\psi_{1a}+e_{it1}>0],\label{biprobit_15}\\ %\qquad
\text{SEPHI}_{it}&= 1[\textbf{x}_{it}'\delta_{2a}+\bar{\textbf{x}}_{i}'\psi_{2a}+e_{it2}>0],\label{biprobit_16} %\qquad
\end{align}
where the subscript $a$ denotes the standardized parameters. The
joint error term, $e_{it}\equiv(e_{it1},e_{it2})$, is assumed to be
independent of $\textbf{x}_{it}$, $\bar{\textbf{x}}_{i}$, and
$\text{SEPHI}_{it}$. Equations (\ref{biprobit_15}) and
(\ref{biprobit_16}) thus satisfy the formulation of a bivariate
probit model, and the log-likelihood function can be easily
constructed using the cumulative bivariate normal distribution
function, $\Phi_{2}\sim\text{Normal(0,$\Sigma$)}$, where $\Sigma$ is
the variance and covariance matrix of $e_{it}$. These scaled
parameters can be consistently estimated using the MLE method.
Finally, assume that $\hat{\delta}_{1a}$, $\hat{\delta}_{Sa}$, and
$\hat{\psi}_{1a}$ are coefficients estimated from the MLE method,
the APE of spouse's EPHI access on own access is
\begin{align}
\Phi[\textbf{x}_{it1}'\hat{\delta}_{1a}+\hat{\delta}_{Sa}+\bar{\textbf{x}}_{i}'\hat{\psi}_{1a}]-\Phi[\textbf{x}_{it1}'\hat{\delta}_{1a}+\bar{\textbf{x}}_{i}'\hat{\psi}_{1a}],
\end{align}
where $\Phi$ is the standard normal CDF.

\subsection{Individual Preference for Employer-Provided Health Insurance Access}\label{section4_2}
As discussed previously, more information is required to estimate
individual preference for spouse's EPHI access, namely
$\Delta_{2}^{k}$ and $\Delta_{3}^{k}$ in the  Eq. (\ref{indirect_1})
\& (\ref{indirect_2}). Since the sample database here does not
contain a direct measure of household member's indirect utility this
study considers income share as a proxy for the estimation. This is
a suitable proxy because husband's and wife's utilities depends on
their relative control over resources, which is widely considered as
the source of family bargaining power (for example, Lundberg and
Pollak, 1996). Since the purpose of this study is to investigate the
individual choices of EPHI access at the household level, the
household income share provides a measure of how the spouse's EPHI
access affects the household member's relative control over
resources.
\par

Consider Eq. (\ref{indirect_1}) \&
(\ref{indirect_2}) with linear unobserved effects: for a
randomly drawn cross section observation $i$,
\begin{align}
\text{U}^{k}_{it}&=
\overline{\text{U}}^{k}_{it}+\delta_{1}^{k}\text{EPHI}_{it}+\delta_{2}^{k}\text{SEPHI}_{it}
+\delta_{3}^{k}\text{EPHI}_{it}\times\text{SEPHI}_{it}+\eta_{i}\label{model_BP},
\end{align}
where $\text{EPHI}_{it}$ and $\text{SEPHI}_{it}$ are binary
variables indicating whether an individual and the individual's
spouse have EPHI access at period $t$, respectively. The term
$\eta_{i}$ is an unobserved heterogeneity that influences EPHI
decisions, such as ability. Similar to the model in Section 2,
$\overline{\text{U}}^{k}_{it}$ denotes the individual $k$'s utility
when neither husband nor wife has EPHI access; $\delta^{k}_{1}$
measures the incremental utility of individual $k$ when he/she
acquires EPHI access; $\delta^{k}_{2}$ measures the incremental
utility for individual $k$ with spousal EPHI access, and
$\delta^{k}_{3}$ measures the utility change when both husband and
wife have EPHI access. To incorporate this theoretical framework
into the proposed econometric model, this study considers the income
share of individual $i$ with sex $k$ at period $t$,
$\text{IS}^{k}_{it}$, as a proxy for person $i$'s indirect utility.
Let $\text{U}^{k}_{it}=\text{IS}^{k}_{it}-u_{it}^{k}$, where
$u^{k}_{it}$ accounts for the proportion of utility that is not
captured by individual $i$'s income share in a linear fashion.
Substituting this equation into Eq.(\ref{model_BP}) yields
\begin{align}
\text{IS}^{k}_{it}&=
\overline{\text{U}}^{k}_{it}+\delta_{1}^{k}\text{EPHI}_{it}+\delta_{2}^{k}\text{SEPHI}_{it}
+\delta_{3}^{k}\text{EPHI}_{it}\times\text{SEPHI}_{it}+\eta_{i}+u_{it}^{k}.\label{model_BP_1}
\end{align}
Assume $\overline{\text{U}}^{k}_{it} $ can be explained by a list of
observed individual characteristics, $X_{it}$, and the idiosyncratic
error, $v_{it}^{k}$, such that
$\overline{\text{U}}^{k}_{it}=X^{k}_{it}\beta^{k}+v_{it}^{k}$,
%%%%%
Eq. (\ref{model_BP_1}) becomes
\begin{align}
  \text{IS}_{it}^{k}&=X^{k}_{it}\beta^{k}+\delta_{1}^{k}\text{EPHI}_{it}+\delta_{2}^{k}\text{SEPHI}_{it}
+\delta_{3}^{k}\text{EPHI}_{it}\times\text{SEPHI}_{it}+\eta_{i}+\varepsilon_{it}^{k},\label{model_13}
\end{align}
where $\varepsilon_{it}^{k}=(u_{it}^{k}+v_{it}^{k})$. Since both
personal and spousal EPHI access may be affected by some unobserved
factors in $\varepsilon_{it}^{k}$, such as time-varying health
insurance premium, this regression consist of three potential
endogenous variables, $\text{EPHI}_{it}$, $\text{SEPHI}_{it}$, and
$\text{EPHI}_{it}\times\text{SEPHI}_{it}$. Estimation will be biased
using the same set of IVs to estimate the coefficients of these
variables simultaneously. The IV approach seems straightforward in
the first but using the same set of IVs to address the endogeneity
problem may generate a bias estimation. The estimation strategy in
this study is to estimate the coefficients of Eq. (\ref{model_13})
using observations conditional on whether individual $i$ has EPHI
access. Specifically, we estimate Eq. (\ref{model_13}) using two
subsamples, say, subsample (1) and subsample (2). The individual $i$
in subsample (1) has no EPHI access ($\text{EPHI}_{it}=0$), while
individual $i$ in subsample (2) has EPHI access
($\text{EPHI}_{it}=1)$. Restricted to subsample (1) and subsample
(2), Eq. (21) becomes
\begin{align}
  \text{IS}_{it}^{k}&=X^{k}_{it}\beta_{0}^{k}+\delta_{2}^{k}\text{SEPHI}_{it}
+\eta_{i0}+\varepsilon_{it0},
\label{model_is1}\\
  \text{IS}_{it}^{k}&=X^{k}_{it}\beta_{1}^{k}+\delta_{1}^{k}+(\delta_{2}^{k}
+\delta_{3}^{k})\text{SEPHI}_{it}+\eta_{i1}+\varepsilon_{it1},
\label{model_is2}
\end{align}
where $\beta_{0}^{k}$ and $\beta_{1}^{k}$ are coefficients to be
estimated using subsamples (1) and (2), respectively. $\eta_{ij}$
and $\varepsilon_{itj}$, where $j=0,1$, stand for unobserved
time-invariant effects and disturbance of the income share
$\text{IS}_{it}^{k}$ in these subsamples. Notice this strategy
deliberately fixes the effect of the individual $i$' EPHI access and
allows the movement of spouse's EPHI decisions, $\text{SEPHI}_{it}$,
to be endogenous in both Eq. (\ref{model_is1}) and Eq.
(\ref{model_is2}). Using the IV estimation, we can thus obtain the
consistent estimates of $\delta^{k}_{2}$ in Eq. (\ref{model_is1})
and $\delta^{k}_{2}$+$\delta^{k}_{3}$ in Eq. (\ref{model_is2}).
However, this strategy is unable to identify  $\delta_{1}^{k}$
because we do not address the endogeneity problem of
$\text{EPHI}_{it}$, and $\delta_{1}^{k}$ is non-separable from the
constant term in $X^{k}_{it}$ in Eq. (\ref{model_is2}).
%%%%%
\par

Since the estimation methods for Eq. (\ref{model_is1}) and
(\ref{model_is2}) are the same, this study uses Eq.
(\ref{model_is1}) to illustrate the proposed estimation methodology.
While the time-constant unobserved effect $\eta_{i}$ might be
correlated with observed explanatory variables, this study applies
the fixed effect approach to estimate the parameters consistently.
The standard fixed-effects approach is to use the time demeaning of
the original equation to remove the individual specific effect
$v_{i}$. Denote the time demeaning notation, $\Delta y_{it}\equiv
y_{it}-\bar{y}$ where $\bar{y}=\frac{1}{T}\sum\limits_{t=1}^{T}
y_{it}$ and apply this to Eq. (\ref{model_13}),  we can obtain
\begin{align}\label{fixmodel}
 \Delta \text{IS}_{it}^{k}&=\Delta X^{k}_{it}\beta_{0}^{k}+\delta_{2}^{k}\Delta \text{SEPHI}_{it}
+\Delta\varepsilon_{it0},\;t=2,...,T.
\end{align}
This fixed-effects transformation cannot rule out the possibility
that $\Delta \text{SEPHI}_{it}$ may still be correlated with $\Delta
\varepsilon_{it0}$. For example, the existence of a time-varying
measurement error in $\Delta \varepsilon_{it0}$ that is related to
$\Delta \text{SEPHI}_{it}$ may still bias the estimates from Eq.
(\ref{fixmodel}). The idiosyncratic assortative match between labor
income and employee benefits like EPHI also makes $\Delta
\text{SEPHI}_{it}$ endogenous. Thus, while the fixed-effects
transformation can effectively eliminate time-invariant unobserved
heterogeneity, the proposed model might still generate an incomplete
solution. To address the issue, this study estimates the equation by
pooled 2SLS using spouse's human capital characteristics, industry group, and firm size as instruments. This choice of instruments
was based on the observation that, while spouse's industry group and
firm size are exogenous to the time-varying unobserved components of
$\Delta \text{IS}_{it}^{k}$, these features are highly correlated
with spousal EPHI access. The following section presents more
details about the IVs.

\section{Results}\label{section5}
The proposed econometric models address the potential endogenous
problem between spousal EPHI access and personal EPHI access and
income share using both the panel data structure and IV methods. As
mentioned in the Introduction, while most of the research on this
topic using either spouse's human capital or job characteristics as
IVs, they might only capture the variation of endogenous variables
in cross section but not in a panel data framework. In order to
avoid the weak instruments problem in panel data models, a set of
instruments should contain both cross section and time dimensional
variations. Human capital variables such as age, education, etc, are
used to tackle down cross section variation. Our solution to the
time dimensional variation is to apply some time-varying job
characteristics of spouse as additional IVs, including the number of
employees and the industry category of the establishment. This
choice of the time-varying instruments is based on the observation
that spouse's industry group and firm size are exogenous to the
time-varying unobserved components in Eq. (\ref{biprobit_15}) \&
(\ref{biprobit_16}) and (\ref{model_is1}) \&
(\ref{model_is2}).\footnote{Currie and Madrian (1999) and Gruber
(2000) link corporate status to EPHI offers.} In addition, firms in
certain industries are more likely to provide EPHI than firms in the
other industries because governments provide different tax subsidies
in different industries. Therefore, these job characteristics are
related to spousal EPHI access but not personal EPHI
access.\footnote{These IVs pass the specification tests suggested by
Angrist and Pischke (2009). In addition to the first-stage results,
both the limited information maximum likelihood estimates and the
reduced form results indicate that our two-stage least squares
estimates are not subject to the weak IV problem. These IVs also
passed the overidentification test. The estimation results are
available upon request from the authors.}
\par
To estimate the effects of spousal EPHI access on personal EPHI
access and income share, this section applies the empirical method
introduced in Section \ref{section4} using data from the MEPS. This
section applies the proposed IVs to examine the effect of spousal
EPHI access on personal EPHI access, which measures the degree of
power conflicts between single and dual EPHI access for a family.
Estimation results show that the effect is negative and
statistically significant, indicating that household members as a
unit may prefer single EPHI access to dual EPHI access on average
due to this conflict. Then, using family income share as a proxy for
individual indirect utility, this section estimates the utility
change of husband and wife separately when their spouses possess
EPHI access. Results show that while both husband's and wife's
utilities decreased when their spouses acquired EPHI access,
husbands experienced greater disutility from their spouse's access
that did the wives. These results suggest that, all else being
equal, having the husband as the single EPHI provider brings the
maximum utility to the household on average.

\subsection{Household Preferences for Employer-Provided Health Insurance Access}
This study investigates the effect of spouse's EPHI access on the
probability of own access to EPHI by estimating Eq.
(\ref{biprobit_15}) and (\ref{biprobit_16}) jointly using the pooled
probit IV estimation presented in the previous section. The control
variables, $\textbf{x}_{it1}$, consist of personal job
characteristics (including employment status, job tenure, wage, and
fringe benefits such as pension plan and sick pay) and personal
human capital characteristics (including age, education, race, and
health status). The term $\textbf{x}_{it1}$ also includes the number
of children in the household younger than 18 years old and regional
unemployment rate and region dummies. As discussed in Section 4,
this study controls time-invariant unobserved effects by applying
Chamberlain's correlated random effect approach, which adds time
averages of all strictly exogenous variables to the regressions.
Spousal human capital variables and firm characteristics serve as
the excluded instruments, $\textbf{x}_{it2}$, to address the
endogenous problem.
\par

\begin{center}
[Table 3 inserts here.]
\end{center}

Tables 3 presents the results from the pooled probit IV regressions.
Empirical evidence indicates that spousal EPHI access reduces the
probability of personal EPHI access. This effect is similar across
gender because both wive's and husbands' regressions estimate
$\Delta_{3}^{h}+\Delta_{3}^{w}$ in  Eq. (\ref{offer_1}) \&
(\ref{offer_2}). These results imply that the husband's and wife's
EPHI access create some power conflicts, and household members as a
whole prefer single to dual EPHI access on average. The APE results
in Table 3 also indicate that an individual is five to six per cent
less likely to find a job with EPHI access if his/her spouse already
has EPHI for the family. Our negative estimate of
$\Delta_{3}^{h}+\Delta_{3}^{w}$ suggests that the family utility
generated from this action ($\Delta_{1}^{k}$+$\Delta_{2}^{l}$) must
be sufficiently large for individual $i$ to find a job with EPHI
access. For example, suppose the husband's bargaining power within
the household is increased by possessing EPHI access. In this case,
the husband's indirect utility increases by $\Delta^{h}_{1}$ while
wife's indirect utility decreases by $\Delta^{w}_{2}$. Now if the
wife also finds a job with EPHI access, it is an optimal strategy
only if her incremental indirect utility ($\Delta^{w}_{1}$) covers
both the loss of husband's incremental disutility ($\Delta^{h}_{2}$)
and the household joint disutility
($\Delta^{h}_{3}$+$\Delta^{w}_{3}$). This empirical evidence
explains why up to 20\% (1404/6920) of the families in the sample
have dual EPHI access.\par

\subsection{Individual Preference for Employer-Provided Health Insurance Access}

While the empirical results in the previous section show that
spousal EPHI access is negatively related to personal EPHI access,
and thus household members as a unit prefers
 single to dual EPHI access, we were unable to identify
 the individual utility change due to spousal EPHI
 access. Therefore, this section uses family income share as
  a proxy of husband's and wife's indirect utility to
  estimate how household members' utility change with
  spouse's EPHI access. This empirical strategy differs
   from the discrete endogenous variable system
  estimated in the previous section. Equations (\ref{biprobit_15}) and (\ref{biprobit_16})
  estimate the discrete choice model by assuming the
  indirect utility as a latent variable, while Eq.(\ref{model_is1})
  and (\ref{model_is2}) directly estimate the indirect
  utility change. Specifically, we consider the total payoffs in Section 2 using the estimates from Eq.(\ref{model_is1}) and (\ref{model_is2}) as follows:
\begin{eqnarray}
\widehat{W}(0,0)& = &X^{h}_{0}\widehat{\beta^{h}_{0}}+X^{w}_{0}\widehat{\beta^{w}_{0}}  \\
\widehat{W}(1,0)& = &X^{h}_{1}\widehat{\beta^{h}_{1}}+X^{w}_{0}\widehat{\beta^{w}_{0}}+\widehat{\delta^{w}_{2}} \\
\widehat{W}(0,1)&  = &X^{h}_{0}\widehat{\beta^{h}_{0}}+X^{w}_{1}\widehat{\beta^{w}_{1}}+\widehat{\delta^{h}_{2}} \\
\widehat{W}(1,1)&  = &X^{h}_{1}\widehat{\beta^{h}_{1}}+X^{w}_{1}\widehat{\beta^{w}_{1}}+\widehat{\delta^{h}_{2}}+\widehat{\delta^{w}_{2}}+\widehat{\delta^{h}_{3}}+\widehat{\delta^{w}_{3}}.
\end{eqnarray}

where the estimates of $\beta^{k}_{0}$ and $\delta^{k}_{2}$ can be identified from Eq. (\ref{model_is1}); and estimates of $\beta^{k}_{1}$ and $\delta^{k}_{3}$ can be identified from Eq. (\ref{model_is2}). The equilibrium conditions of this game are thus as follows:
\begin{align}
\text{a}^{h}&= 1[X^{h}_{1}\widehat{\beta^{h}_{1}}-X^{h}_{0}\widehat{\beta^{h}_{0}}+\widehat{\delta^{w}_{2}}+\text{a}^{w}(\widehat{\delta_{3}^{h}}+\widehat{\delta_{3}^{w}})\geq 0]\label{model_IS4}\\
\text{a}^{w}&=
1[X^{w}_{1}\widehat{\beta^{w}_{1}}-X^{w}_{0}\widehat{\beta^{w}_{0}}+\widehat{\delta^{h}_{2}}+\text{a}^{h}(\widehat{\delta_{3}^{h}}+\widehat{\delta_{3}^{w}})\geq
0]\label{model_IS5}.
\end{align}
These conditions provide another source of identification for Eq. (\ref{model_13}).  The difference between $X^{k}_{1}\widehat{\beta^{k}_{1}}$ and $X^{k}_{0}\widehat{\beta^{k}_{0}}$ provides an estimate of $\delta^{k}_{1}$ with the assumption that, after controlling the effects from the observables in $ X^{k}_{it} $ and the spouse's EPHI access, personal EPHI access is the only source of variation that explains the difference between the predicted income share of households with and without personal EPHI access. Therefore, Eq. (\ref{model_IS4}) and (\ref{model_IS5}) can
be regarded as the empirical counterparts of  Eq. (\ref{oya_1}) and
(\ref{oya_2}).
\par
To regress the income share equation Eq. (\ref{model_is1}) and
(\ref{model_is2}), this study applies the same control variables as those used
for Eq. (\ref{biprobit_15}) and (\ref{biprobit_16}), and controls the
time-invariant unobserved effects $v_{i}$ by applying the
fixed-effects method in Eq. (\ref{fixmodel}). In addition, since
age, race, and education remained constant during the sample period,
these variables may not be identified or distinguished from fixed
effects. Therefore, these variables interact with the round dummies
to control their effects on the dependent variable, which might
change over time. Endogenous problem of $\Delta \text{SEPHI}_{it}$ in Eq. (\ref{fixmodel}) is addressed
by applying the pooled 2SLS method using spouse's human capital
variables and firm characteristics as the IVs.
\par

\begin{center}
[Table 4 inserts here.]
\end{center}

Table 4 presents the empirical results. As predicted in Section 2,
spousal access to EPHI reduces both husband's and wife's income
share regardless of whether personal EPHI is offered. In particular,
columns (1) and (3) in Table 4 show that, without personal EPHI
access, the husband's access to EPHI reduces the wife's income share
by 12\% ($\widehat{\delta^{w}_{2}}$), while the wife's access to
EPHI reduces the husband's share by 18\%
($\widehat{\delta^{h}_{2}}$). These estimates suggest that husbands'
utility decreases more than that of the wife after their spouse
became the single EPHI provider for the family. These results
suggest a traditional spousal relationship in which the husbands'
bargaining power is affected more by the wife's EPHI access than
vice versa.\footnote{Asymmetric effects across gender also appear in
Lundberg (1988) and Pencavel (1998), where the effects from spouse's
employment or human capital decisions on personal labor market
outcomes are also asymmetrically distributed across gender. Both
studies explain these results using the family power balance
explanation. Royalty and Abraham (2006) also found an asymmetric
effect from spousal EPHI access on personal working hours. They
argued that while husbands work more when their wives have a good
job due to positive assortative mating,
 wives work
less when their husbands have a good job because the income effect dominates the assortative mating effect for the wives.}
\par

When the husband or wife has an EPHI offer, columns (2) and (4) in
Table 4 indicate that the husbands' income share decreases more than
wives' share after the other spouse obtains EPHI access. As shown in
Eq. (\ref{model_is1}) and  (\ref{model_is2}), the husband's
disutility from dual EPHI access ($\widehat{\delta^{h}_{3}}$) is
-.243, while the wife's disutility from dual EPHI access
($\widehat{\delta^{w}_{3}}$) is -.055. These results are similar to
the estimates of the bivariate probit model, where having dual EPHI
access reduces the household joint utility. Thus, the husband's and
wife's choice criteria in Eq. (\ref{model_IS4}) and
(\ref{model_IS5}) become
\begin{align}
\text{a}^{h}&= 1[X^{h}_{1}\widehat{\beta^{h}_{1}}-X^{h}_{0}\widehat{\beta^{h}_{0}}-.184-\text{a}^{w}(.298)\geq 0]\\
\text{a}^{w}&=
1[X^{w}_{1}\widehat{\beta^{w}_{1}}-X^{w}_{0}\widehat{\beta^{w}_{0}}-.123-\text{a}^{h}(.298)\geq
0].
\end{align}
\par

\begin{center}
[Figure 1 inserts here.]
\end{center}
Figure 1 graphs these interaction effects, where the individual
utility from holding EPHI is denoted as $\widehat{\delta^{h}_{1}}$
for the husband and $\widehat{\delta^{w}_{1}}$ for the wife.
Since
the wife's EPHI access significantly reduces the husband's indirect
utility and therefore the joint household utility, more families
should have the husband as the single EPHI provider. Specifically,
when the individual utility gain from having EPHI access ($\delta^{k}_{1}$) is too
small to compensate for the spouse's loss
($\delta^{k}_{2}$), the optimal strategy is to choose a
job without EPHI access regardless of whether the spouse has EPHI
access. In this case, our estimates suggest that the wife's
threshold level to choose this strategy is higher (.184) than the
husband's (.123), and wives are therefore more likely to apply this
strategy. On the other hand, when the individual utility gain from
having EPHI access is large enough to cover both the spouse's and
the joint utility loss, the optimal strategy is to choose a job with
EPHI regardless of spousal EPHI access availability. Our estimates suggest that this strategy is more likely to become the
husband's dominant strategy, because the husband's threshold level
(.421) is smaller than the wife's threshold level (.482).
Consequently, the empirical results explain why 41\% (2806/6920) of
the families in the sample have the husbands as the single EPHI
provider. On the other hand, only 15\% (1047/6920) of the families
in the sample have the wives as the sole EPHI provider.
\par


\section{Conclusion}\label{section6}
This study provides a new perspective and framework showing why some
families have single EPHI access while some have dual EPHI access.
We investigate household preferences for EPHI access using the
empirical methodology proposed by Bresnahan and Reiss (1991), and
model the household joint health insurance decision as a
cooperative, simultaneous game. We estimate the effects of spousal
EPHI access on personal access, predict how husband and wife make
their decisions based on household joint utility and family power balance. Results show that
an average household generally prefers single to dual EPHI access,
suggesting that the husband's and wife's EPHI access are not
complementary due to the bargaining power conflicts. Using household
income share as a proxy for the individual indirect utility, this
study shows that husbands have greater incentive to acquire EPHI
access than their wives, because wives' EPHI access decreases the
husbands' utility more than vice versa. These estimates suggest a
traditional spousal relationship in which husbands may possess more
bargaining power in an average household.
\par

The main challenge of empirical research on this topic comes from
the endogeneity of EPHI access. The empirical approach in this study
includes several novel concepts. First, this study uses the panel
data structure of the MEPS data to eliminate permanent unobserved
effects, i.e., it applies a fixed-effects transformation to remove
the unobserved permanent effects of EPHI offers. Second, two types
of instruments are used in the transformed empirical model of EPHI:
human capital variables and spouse's firm characteristics.
\par

This paper highlights an important link between marital gender power
and EPHI offers. The gender difference in response to spousal EPHI
access in terms of income share and EPHI access suggests that
husbands and wives face different opportunity costs in the labor
market when their spouses already have EPHI access. Therefore,
husbands and wives behave differently to maximize their personal
utilities and their family well-being. There are two implications
from our results: first, as more women obtain EPHI access from their
labor market outcomes, they may become the sole EPHI provider and
their marital bargaining power will rise, resulting in greater
equality in the family. Viewed from this perspective, dual EPHI
access should be more prevalent in married household in the future.
Second, since single EPHI access may be sufficient for some
families, individuals might look for jobs without EHPI access if
their spouse already has EPHI access. For example, if there is a
trade-off between wages and EPHI access, some people may try to find
high-paying jobs without EPHI. However, this situation can also
create job lock for the sole EPHI provider, and the current system
of employer-based insurance may produce larger welfare losses.
\par

\begin{thebibliography}{12}

\bibitem{1} Abraham, J.M., Royalty, A.B., 2005. Does Having Two Earners in the Household Matter for Understanding How
                Well Employer-Based Health Insurance Works? \emph{Medical Care Research and Review} 62 (2), pp. 167-186.

\bibitem{1} Abraham, J.M., Vogt, W.B., Gaynor, M.S. 2002. Household Demand for Employer-Based Health Insurance, \emph{NBER Working Paper,} W9144.


\bibitem{2} Angrist, J., Pischke J., 2009. \emph{Mostly Harmless Econometrics}. Princeton University Press, Princeton, New Jersey.


\bibitem{1} Bresnahan T.F. and Reiss, P.C., 1991, Empirical Models of Discrete Games, \emph{Journal of Econometrics} 48, 57-81.


%\bibitem{1} Agarwal, B., 1997.
%                 Bargaining and gender relations: within and beyond the household.
 %                Feminist Economics 3 (1), 1-51.



%\bibitem{3} Aura, S., 2005. Does the balance of power within a family matter? The case of the Retirement Equity Act.
%                  Journal of Public Economics 89 (9-10), 1699-1717.

%\bibitem{4} Basu, K., 2006.
 %                 Gender and Say: a model of household behaviour with endogenously determined balance of power.
    %              Economic Journal 116 (511), 558-580.

%\bibitem{5} Becker, G., 1974.
%                 A theory of social interactions.
%                 Journal of Political Economy 82 (6), 1063-1094.

%\bibitem{6} Beegle, K., Frankenberg, E., Thomas, D., 2001.
%                   Bargaining power within couples and use of prenatal and delivery care in Indonesia.
%                   Studies in Family Planning 32 (2), 130-146.

\bibitem{7} Bhargavan, M., 2000.
               The Effect of Employer-Provided Health Insurance on Labor Market Participation.
               PhD Dissertation, Rutgers University.

%\bibitem{8} Blumberg, R.L., 1988.
%        Income under female versus male control: hypotheses from a theory of gender stratification and data from the third world.
%                Journal of Family Issues 9 (1), 51-84.

%\bibitem{9} Blumberg, R.L., Coleman, M.T., 1989.
%             A theoretical look at the gender balance of power in the American couple.
%             Journal of Family Issues 10 (2), 225-250.

%\bibitem{10} Buchmueller, T.C., Lettau, M.K., 1997.
%              Estimating the wage-health insurance trade-off: more data problems?
%              Working Paper, University of California at Irvine.

%\bibitem{11} Buchmueller, T.C., Valletta, R.G., 1998.
%                 Health Insurance and the U.S. Labor Market.
%                 FRBSF Economic Letter, Federal Reserve Bank of San Francisco:
%                 http://www.frbsf.org/econrsrch/wklyltr/wklyltr98/el98-12.html

\bibitem{12} Buchmueller, T.C., Valletta, R.G., 1998.
                  The Effect of Health Insurance on Married Female Labor Supply.
                  \emph{Journal of Human Resources} 34 (1), pp. 42-70.

\bibitem{13} Chamberlain, G., 1982. Multivariate Regression Models for Panel Data. \emph{Journal of Econometrics} 18 (1), pp. 5-46.

\bibitem{14} Chamberlain, G., 1984. Panel Data. in: Griliches, Z., Intriligator, M.D. (Ed.)
                  \emph{Handbook of Econometrics}, Volume 2. North Holland, Amsterdam, pp. 1247-1318.

\bibitem{15} Currie, J., Madrian, B.C., 1999.
             Health, Health Insurance and the Labor Market.
             in: Ashenfelter, O., Card, D. (Ed.) \emph{Handbook of Labor Economics}. North-Holland, Amsterdam, pp. 3309-3416.
\bibitem{15} Dey, M. S. and Flinn, C. J., 2005.
             An Equilibrium Model of Health Insurance
             Provision and Wage Determination. \emph{Econometrica}, 73(2), pp. 571-627.
\bibitem{15} Dey, M. S. and Flinn, C. J., 2008.
             Household Search and Health Insurance Coverage.
             \emph{Journal of Econometrics}, 145(1-2), pp. 43-63.
%\bibitem{16} Due, J.M., Gladwin, C.H., 1991.
%                  Impact of structural adjustment programs on African women farmers and female headed households
%                  American Journal of Agricultural Economics 73 (5), 1431-1439.

%\bibitem{17} Duflo, E., 2000. Grandmothers and granddaughters: Old age pension and intra-household allocation in South Africa.
%                 NBER Working Paper No. 8061.

%\bibitem{18} England, P., Kilbourne, B.S., 1990.
%                  Markets, marriages, and other mates: the problem of power.
%                  in: Friedland, R., Robertson, A.F. (Ed.) Beyond the Marketplace: Rethinking Economy and Society.
%                 Aldine de Gruyter, New York, pp. 163-89.

%\bibitem{19} Ferber, R., 1971.
%                 Family decision making and economic behavior: a review.
%                 in: Sheldon, E.B. (Ed.) Family Economic Behavior: Problems and Prospects.
%                 J. B. Lippincott Co., Philadelphia.

%\bibitem{20} Friedberg, L., Webb, A., 2006.
%                 Determinants and consequences of bargaining power in households.
%                 NBER Working Paper No. 12367.

\bibitem{21} Gentry, W.M., Peress, E., 1994.
                  Taxes and Fringe Benefits Offered by Employers.
                  \textit{NBER Working Paper}, No. 4764.

%\bibitem{22} Gladwin, C.H., McMillan, D., 1989.
%                  Is a turnaround in Africa possible without helping African women to farm?
%                  Economic Development and Cultural Change 37 (2), 345-369.

%\bibitem{23} Grossman, M., 1972.
%               On the concept of health capital and the demand for health.
%               Journal of Political Economy 80 (2), 223-255.
%
%\bibitem{24} Grossman, M., 2000.
%                The human capital model of the demand for health.
%                in: Culyer A.J., Newhouse, J. (Ed.) Handbook of Health Economics. North-Holland, Amsterdam, pp. 347-408.

\bibitem{25} Gruber, J., 2000.
                  Health Insurance and the Labor Market.
                 in: Culyer A.J., Newhouse, J. (Ed.) \emph{Handbook of Health Economics}. North-Holland, Amsterdam, pp. 3309-3416.

\bibitem{26} Gruber, J., Poterba, J., 1996.
               Tax Subsidies to Employer-Provided Health Insurance.
               in: Feldstein M., Poterba, J.M. (Ed.) \emph{Empirical Foundations of Household Taxation}.
               Univeristy of Chicago Press, Chicago, pp. 135-164.

%\bibitem{27} Haddad, L., Hoddinott, J., Alderman, H., (Ed.) 1997.
%                  Intra-household Resource Allocation in Developing Countries: Models, Methods and Policy.
%                  The Johns Hopkins University Press, Baltimore,  pp. 112-128.
%
%\bibitem{28} Hoddinott, J., Haddad, L., 1995. Does female income share influence household expenditures?
%                  Evidence from C\^{o}te d'Ivoire. Oxford Bulletin of Economics and Statistics 57 (1), pp. 77-96.

\bibitem{29} Honig, M., Dushi, I., 2005.
                 Household Demand for Health Insurance: Price and Spouse's Coverage.
                 Working Paper, Hunter College.

\bibitem{29} Kooreman, P. and Wunderink S., 1997, \emph{The Economics of Household Behaviour}, New York : St. Martin's Press.
%\bibitem{30} Leibowitz, A., 1983.
%                  Fringe benefits in employee compensation.
%                  in: The Measurement of Labor Cost. University of Chicago Press, Chicago, pp. 371-394.

\bibitem{31} Lundberg, S., 1988.
                  Labor Supply of Husbands and Wives: A Simultaneous Equations Approach.
                  \textit{Review of Economics and Statistics}, 70 (2), 224-235.

\bibitem{32} Lundberg, S., Pollak, R.A., 1996.
                 Bargaining and Distribution in Marriage.
                 \emph{Journal of Economic Perspectives} 10 (4), 139-158.

%\bibitem{33} Lundberg, S., Pollak, R.A., 2003.
%                  Efficiency in Marriage.
%                  Review of Economics of the Household, 1 (3),  153-167.

%\bibitem{34} Lundberg, S., Pollak, R.A., Wales, T.J., 1997.
%                  Do husbands and wives pool their resources? evidence from the United Kingdom child benefit.
%                  Journal of Human Resources 32 (3), 463-480.

%\bibitem{35} Lundberg, S., Startz R., Stillman, S., 2003.
%                 The retirement consumption puzzle: a marital bargaining approach.
%                Journal of Public Economics, 87 (5-6), 1199-1218.

%\bibitem{36} Manser, M., Brown, M., 1980.
%                 Marriage and household decision-making: a bargaining analysis.
%                 International Economic Review 21 (1), 31-44.

\bibitem{37} McElroy, M.B., 1990.
                 The Empirical Content of Nash-Bargained Household Behavior.
                \emph{Journal of Human Resources} 25 (4), 559-583.

%\bibitem{38} McElroy, M.B., Horney, M.J., 1981.
%                  Nash-bargained household decisions: toward a generalization of the theory of demand.
%                  International Economic Review 22 (2), 333-349.

\bibitem{39} Marquis, M.S., Kapur, K., 2004.  Family Decision Making When Two Workers are Offered Group Coverage.
                  Unpublished manuscript, University College Dublin

%\bibitem{40} Miller Jr., R.D., 2004.
%                  Estimating compensating differentials for employer-provided health insurance benefits.
%                  International Journal of Health Care Finance and Economics 4 (1), 27-41.

%\bibitem{41} Monheit, A.C., 1985.
%                 The employed uninsured and the role of public policy. National health care expenditures study.
%                  Inquiry 22, pp. 348-364.

\bibitem{42} Monheit, A.C., Schone, B.S., Taylor, A.K., 1999. Health Insurance Choices in Two-Worker Households:
                  Determinants of Double Coverage. \emph{Inquiry} 36, pp. 12-29.


\bibitem{43} Olson, C.A., 1998.
         A Comparison of Parametric and Semiparametric Estimates of the Effect of Spousal Health Insurance Coverage on
         Weekly Hours Worked by Wives.
             \emph{Journal of Applied Econometrics} 13 (5), 543-565.

\bibitem{44} Olson, C.A., 2000.
                  Part-Time Work, Health Insurance Coverage, and the Wages of Married Women.
                 in: Alpert, W.T., Woodbury, S.A. (Ed.) \emph{Employee Benefits and Labor Markets in Canada and the United States}.
                 W. E. Upjohn Institute for Employment Research, Kalamazoo, Michigan, pp. 295-324.

\bibitem{45} Pencavel, J., 1998.
                 Assortative Mating by Schooling and the Work Behavior of Wives and Husbands.
                 \emph{American Economic Review} 88 (2), 326-329.

%\bibitem{46} Phipps, S.A., Burton, P.S., 1998.
%                  What's mine is yours? The influence of male and female incomes on patterns of household expenditure.
%                  Economica 65, 599-613.

%\bibitem{47} Pollak, R.A., 2005.
%                  Bargaining power in marriage: earning, wage rates and household production.
%                  NBER Working Paper No. 11239.
\bibitem{47} Reiss P. C. and Wolak F. A., 2007, Structural Econometric Modeling: Rationales and
Examples from Industrial Organization, in J. J. Heckman and E. E. Leamer (eds.),
Handbook of Econometrics, Volume 6A, Chapter 64, North Holland, pp. 4277-4415.

\bibitem{47} Royalty, A.B., 2000, Tax Preferences for Fringe Benefits and Workers' Eligibility for
 Employer Health Insurance, \textit{Journal of Public Economics}, 72(2), 209—227.


\bibitem{48} Royalty, A.B., Abraham, J.M., 2006.
                  Health Insurance and Labor Market Outcomes: Joint Decision-Making Within Households.
                  \emph{Journal of Public Economics} 90 (8-9), 1561-1577.

%\bibitem{49} Ryan, S., 1997.
%                 Employer-provided health insurance and compensating wage differentials: evidence from the survey of
%                 income and program participation.
%                 Working Paper, University of Missouri at Columbia.

\bibitem{50} Schone, B.S., Vistnes, J.P., 2001.
                 The Relationship Between Health Insurance and Labor Force Decisions: An Analysis of Married Women.
                 Working Paper, Agency for Healthcare Research and Quality, U.S. Department of Health and Human Services.

%\bibitem{51} Sprey, J., 1972.
%                 Family power structure: a critical comment.
%                 Journal of Marriage and the Family 34 (2), 235-238.
%
%\bibitem{52} Turk, J.L., Bell, N.W., 1972.
%                  Measuring power in families.
%                  Journal of Marriage and the Family 34 (2), 215-223.
%
%\bibitem{53} Turner, R.H., 1970.
%                   Family Interaction. Wiley $\&$ Sons, New York.

\bibitem{54} Wellington, A.J., Cobb-Clark, D.A., 2000.
                 The Labor-Supply Effects of Universal Health Coverage: What Can We Learn From Individuals With Spousal Coverage.
                  in: Polachek S. (Ed.) \emph{Research in Labor Economics}, vol. 19. Elsevier Science, Amsterdam, pp. 315-344.

\bibitem{55} Woodbury, S.A., Huang, W., 1991.
                  \emph{The Tax Treatment of Fringe Benefits}. W. E. Upjohn Institute for Employment Research, Kalamazoo, Michigan.

%\bibitem{56} Wooldridge, J., 1995.
%                  Selection corrections for panel data models under conditional mean independence assumptions.
%                  Journal of Econometrics 68 (1), 115-132.

\bibitem{57} Wooldridge, J., 2010. \emph{Econometric Analysis of Cross Section and Panel Data}.
                 The MIT press, Cambridge, Massachusetts.

\end{thebibliography}



\newpage \extrarowheight=-7pt
\begin{longtable}{@{}l@{\hspace{-8pt}}l}\label{Table_1}\\[-2pt]
\caption{Variables Definition}\\[-2pt]
\toprule
Variable& Definition\\
\hline\\[-6pt]
\textbf{Proxy Variables}\\[-2pt]
\hspace{0.5cm}Income share&Personal income divided by total personal
incomes\\
\textbf{Health Insurance Variables}\\[-2pt]
\hspace{0.5cm}Husband's offer&Dummy variable =1 if husband has an
offer
of \\[-6pt]
\hspace{0.5cm}&employer health insurance, =0 otherwise \\[-2pt]
\hspace{0.5cm}Wife's offer&Dummy variable =1 if husband has an offer
of \\[-6pt]&employer health insurance, =0 otherwise \\[-6pt]
\textbf{Human Capital Characteristics}\\[-2pt]
\hspace{0.5cm}Age&Age of a person\\[-2pt]
\hspace{0.5cm}Education&Number of years of education of a person\\[-2pt]
\hspace{0.5cm}Non-white&Dummy variable =1 if a person is non-white,\\[-6pt]
& =0 otherwise\\[-2pt]
\hspace{0.5cm}Perceived health&Dummy variable =1 if a person is in good health  \\[-6pt]
\hspace{0.5cm}status &status,=0 otherwise\\[-1pt]
\textbf{Income and Job Characteristics}\\[-2pt]
\hspace{0.5cm}Personal income& The sum of all personal income components  \\[-3pt]
\hspace{0.5cm}Wage& Hourly wage for the current main job \\[-3pt]
\hspace{0.5cm}Union&Dummy variable =1 if a person belongs to a labor \\[-5pt] &union, =0 otherwise \\[-3pt]
\hspace{0.5cm}Pension&Dummy variable =1 if a person has a pension plan\\[-5pt] &through employer, =0 otherwise  \\[-3pt]
\hspace{0.5cm}Sick pay&Dummy variable =1 if a person paid sick leave\\[-5pt] &through employer, =0 otherwise  \\[-3pt]
\hspace{0.5cm}Tenure&The term during the current main job \\[-3pt]
\hspace{0.5cm}Employed&Dummy variable =1 if a person is employed,\\[-5pt] &=0 otherwise  \\[-1pt]
\textbf{Family Characteristics}\\[-2pt]
\hspace{0.5cm}Kid0-18&Number of children in the household under 18\\[-2pt]
\hspace{0.5cm}Regional unemp. rate (\%)&Unemployment rate for region in which the \\[-3pt]&household resides\\[-1pt]
\textbf{Region Dummy Variables}\\[-2pt]
\hspace{0.5cm}Northeast&Dummy variable =1 if a person resides in Northeast\\[-5pt] & Census Region, =0 otherwise \\[-3pt]
\hspace{0.5cm}Midwest&Dummy variable =1 if a person resides in Midwest\\[-5pt] &Census Region, =0 otherwise \\[-3pt]
\hspace{0.5cm}South&Dummy variable =1 if a person resides in South\\[-5pt] &Census Region, =0 otherwise \\[-3pt]
\hspace{0.5cm}West&Dummy variable =1 if a person resides in West\\[-5pt] &Census Region, =0 otherwise \\[-3pt]
\textbf{Instruments Variables$^{a}$}\\[-2pt]
\hspace{0.5cm}Firm size &The number of employees at the person's
current
\\[-5pt]
& main job\\[-1pt]
\hspace{0.5cm}Industry Group&The industry codes for a person's
current
main job \\[-1pt]
\bottomrule \multicolumn{2}{l}{Note: a. For people not employed, the values of these IV are zeros.}\\[-8pt]
\end{longtable}




\newpage
\extrarowheight=-8pt
\begin{longtable}{@{}l@{}.{3}c@{}c@{}c@{}c@{}}\label{Table_2}\\
\caption{Sample Statistics by Family EPHI Status}\\
\toprule
&\multicolumn{1}{r}{Total}&Both&None&Wife&Husband\\[-2pt]
\multicolumn{1}{c}{Variables}&&&&Only&Only\\[-2pt]
\midrule
\endfirsthead
\toprule
\multicolumn{1}{c}{Variables}&\multicolumn{1}{r}{Total}&Both&None&Wife&Husband\\[-2pt]
&&&&Only&Only\\[-2pt]
\midrule
\endhead
\multicolumn{6}{r}{(continued)}
\endfoot
\endlastfoot
\multicolumn{2}{c}{\textbf{Household Income }}  &&&&\\[-2pt]
\multicolumn{2}{l}{\textit{Personal Income (in \$1,000)}}&&&&\\[-2pt]
\hspace{0.5cm}Husband&43.058&47.791&24.229&34.578&55.006\\[-2pt]
&(34.965)&(31.021)&(24.389)&(27.840)&(38.759)\\
\hspace{0.5cm}Wife&26.815&42.761&12.462&45.172&20.468\\[-2pt]
&(28.691)&(26.275)&(17.662)&(30.646)&(27.103)\\[-2pt]
\multicolumn{2}{l}{\textit{Household Income Share}}&&&&\\[-2pt]
\hspace{0.5cm}Husband& 0.636&0.524&0.667&0.418&0.756\\[-2pt]
&\hspace{4mm}(0.272)&(0.140)&(0.313)&(0.209)&(0.245)\\[-2pt]
\hspace{0.5cm}Wife&0.359&0.476&0.314&0.582&0.244\\[-2pt]
&\hspace{2mm}(0.270)&(0.140)&(0.302)&(0.209)&(0.245)\\[-2pt]
\multicolumn{2}{c}{\textbf{Human Capital}}  &&&&\\[-4pt]
\multicolumn{2}{c}{\textbf{Characteristics}} &&&&\\[-2pt]
\multicolumn{2}{l}{\textit{Age}}&&&&\\[-2pt]
\hspace{0.5cm}Husband&44.010&43.763&43.816&45.450&43.710\\[-2pt]
&\hspace{2mm}(10.029)&(9.433)&(11.345)&(10.168)&(9.374)\\[-2pt]
\hspace{0.5cm}Wife&41.821&41.632&41.331&43.359&41.632\\[-2pt]
&\hspace{4mm}(9.950)&(9.298)&(11.411)&(9.384)&(9.482)\\
\multicolumn{2}{l}{\textit{Education}}&&&&\\[-2pt]
\hspace{0.5cm}Husband&12.645&13.737&10.594&13.139&13.129\\[-2pt]
&\hspace{4mm}(3.410)&(2.568)&(3.873)&(2.846)&(3.172)\\[-2pt]
\hspace{0.5cm}Wife&12.720&14.099&10.619&13.617&12.939\\[-2pt]
&\hspace{4mm}(3.463)&(2.521)&(3.780)&(2.957)&(3.280)\\
\multicolumn{2}{l}{\textit{Non-white}}&&&&\\[-2pt]
\hspace{0.5cm}Husband&0.177&0.249&0.134&0.241&0.142\\[-2pt]
&\hspace{4mm}(0.382)&(0.433)&(0.341)&(0.428)&(0.349)\\[-2pt]
\hspace{0.5cm}Wife&0.181&0.263&0.138&0.229&0.146\\[-2pt]
&\hspace{4mm}(0.385)&(0.441)&(0.345)&(0.421)&(0.353)\\
\multicolumn{3}{l}{\textit{Perceived health status}}&&&\\[-2pt]
\hspace{0.5cm}Husband&0.600&0.658&0.492&0.566&0.647\\[-2pt]
&\hspace{4mm}(0.490)&(0.475)&(0.500)&(0.496)&(0.478)\\[-2pt]
\hspace{0.5cm}Wife&0.586&0.641&0.467&0.634&0.611\\[-2pt]
&\hspace{4mm}(0.493)&(0.480)&(0.499)&(0.482)&(0.488)\\
\multicolumn{2}{c}{\textbf{Job Characteristics}}  &&&&\\[-4pt]
\multicolumn{2}{l}{\textit{Employed}}&&&&\\[-2pt]
\hspace{0.5cm}Husband&0.876&1&0.670&0.707&1\\[-2pt]
&\hspace{4mm}(0.329)&(0)&(0.470)&(0.455)&(0)\\[-2pt]
\hspace{0.5cm}Wife&0.630&1&0.365&1&0.465\\[-2pt]
&(0.483)&(0)&(0.482)&(0)&(0.499)\\
\multicolumn{2}{l}{\textit{Tenure}}&&&&\\[-2pt]
\hspace{0.5cm}Husband&3.417&5.461&1.440&1.635&4.228\\[-2pt]
&\hspace{4mm}(6.922)&(8.284)&(3.766)&(4.384)&(7.873)\\[-2pt]
\hspace{0.5cm}Wife&2.071&4.548&0.605&3.588&1.129\\[-2pt]
&\hspace{4mm}(5.132)&(6.920)&(2.378)&(6.964)&(3.665)\\
\multicolumn{2}{l}{\textit{Hourly Wage}}&&&&\\[-2pt]
\hspace{0.5cm}Husband& 18.882&23.141&8.682&14.513&24.418\\[-2pt]
&(14.796)&(12.746)&(9.149)&(13.500)&(15.256)\\
\hspace{0.5cm}Wife&10.822&19.203&3.866&20.069&7.287\\[-2pt]
&(12.285)&(10.798)&(6.641)&(12.968)&(10.808)\\
%\multicolumn{2}{l}{Union}&&&&\\[-2pt]
%\hspace{0.5cm}Husband&0.062&0.116&0.008&0.017&0.083\\[-2pt]
%&\hspace{4mm}(0.241)&(0.321)&(0.091)&(0.130)&(0.277)\\[-2pt]
%\hspace{0.5cm}Wife&0.034&0.080&0.010&0.076&0.010\\[-2pt]
%&\hspace{4mm}(0.181)&(0.271)&(0.101)&(0.264)&(0.098)\\
\multicolumn{2}{l}{\textit{Pension Plan}}&&&&\\[-2pt]
\hspace{0.5cm}Husband&0.513&0.737&0.088&0.326&0.723\\[-2pt]
&\hspace{4mm}(0.500)&(0.441)&(0.284)&(0.469)&(0.447)\\[-2pt]
\hspace{0.5cm}Wife&0.357&0.726&0.060&0.729&0.208\\[-2pt]
&\hspace{4mm}(0.479)&(0.446)&(0.238)&(0.444)&(0.406)\\
\multicolumn{2}{l}{\textit{Sick Pay}}&&&&\\[-2pt]
\hspace{0.5cm}Husband&0.559&0.745&0.175&0.403&0.752\\[-2pt]
&\hspace{4mm}(0.497)&(0.436)&(0.380)&(0.491)&(0.432)\\[-2pt]
\hspace{0.5cm}Wife&0.418&0.826&0.090&0.850&0.248\\[-2pt]
&\hspace{4mm}(0.493)&(0.379)&(0.286)&(0.357)&(0.432)\\
\multicolumn{2}{c}{\textbf{Family Characteristics}} &&&&\\[-2pt]

\multicolumn{1}{l}{\textit{Kid0-18}}&\hspace{4mm}1.547&1.219&1.791&1.237&1.682\\[-2pt]
&\hspace{4mm}(1.379)&(1.208)&(1.479)&(1.252)&(1.391)\\[-2pt]
\multicolumn{1}{l}{\textit{Regional unemp.}}&\hspace{4mm}4.620&4.670&4.608&4.614&4.604\\[-2pt]
\multicolumn{1}{l}{\hspace{2mm}\textit{rate} (\%)}&\hspace{4mm}(0.336)&(0.340)&(0.356)&(0.323)&(0.325)\\[-2pt]
\multicolumn{2}{c}{\textbf{Geographical}} &&&&\\[-2pt]
\multicolumn{2}{c}{\textbf{Distribution}} &&&&\\[-2pt]

\multicolumn{1}{l}{\textit{Northeast}}&\hspace{4mm}0.160&0.163&0.173&0.206&0.173\\[-2pt]
&\hspace{4mm}(0.367)&(0.369)&(0.311)&(0.404)&(0.378)\\
\multicolumn{1}{l}{\textit{Midwest}}&\hspace{4mm}0.205&0.229&0.126&0.226&0.231\\[-2pt]
&\hspace{4mm}(0.403)&(0.421)&(0.332)&(0.418)&(0.421)\\
\multicolumn{1}{l}{\textit{South}}&\hspace{4mm}0.362&0.363&0.409&0.362&0.333\\[-2pt]
&\hspace{4mm}(0.481)&(0.481)&(0.492)&(0.481)&(0.471)\\
\multicolumn{1}{l}{\textit{West}}&\hspace{4mm}0.273&0.245&0.357&0.207&0.263\\[-2pt]
&\hspace{4mm}(0.446)&(0.430)&(0.479)&(0.405)&(0.440)\\
\multicolumn{2}{c}{\textbf{Instruments}}  &&&&\\[-4pt]
\multicolumn{2}{c}{\textbf{Variables}}  &&&&\\[-2pt]
\multicolumn{2}{l}{\textit{Firm size}}&&&&\\[-2pt]
\hspace{0.5cm}Husband&144.271&189.931&52.9425&82.300&198.594\\[-2pt]
&\hspace{2mm}(178.500)&(185.914)&(111.169)&(145.174)&(189.052)\\
\hspace{0.5cm}Wife&100.705&189.273&28.898&207.897&58.806\\[-2pt]
&\hspace{2mm}(168.073)&(197.240)&(85.373)&(197.093)&(135.493)\\[-2pt]
\hline
Number of &&&&&\\[-4pt]
observations&\multicolumn{1}{r}{6920}&\multicolumn{1}{r}{1408}&\multicolumn{1}{r}{1662}&\multicolumn{1}{r}{1046}&\multicolumn{1}{r}{2804}\\
\bottomrule
\end{longtable}


%\newpage
%\begin{longtable}{@{}lcccccc@{}}\label{Table_3}\\
%\caption{First-stage results}\\
%%\multicolumn{3}{l}{Endogenous variables (Spouse offer)}\\
%\toprule &\multicolumn{2}{c}{Female share
%equation}&\multicolumn{2}{c}{Male share
%equation}&\multicolumn{2}{c}{Offer
%equation}\\[-3pt]
% &\multicolumn{2}{c}{endogenous
%variables}&\multicolumn{2}{c}{endogenous
%variables}&\multicolumn{2}{c}{endogenous variables}\\[-3pt]
%\cmidrule(l){2-3}\cmidrule(l){4-5}\cmidrule(l){6-7}
%&Subsample&Subsample&Subsample&Subsample&Female&Male\\[-3pt]
%&(1)&(2)&(3)&(4)&sample&sample\\[-3pt]
%&Husband&Husband&Wive&Wive&Husband&Wive\\[-3pt]
%Instruments&offer&offer&offer&offer&offer&offer\\[-3pt]
%\midrule
%Firm size&0.001&0.001&0.001&0.000&0.003&0.002\\[-3pt]
%&(0.000)&(0.000)&(0.000)&(0.000)&(0.001)&(0.000)\\
%Industry&0.032&0.043&0.015&0.032&0.115&0.101\\[-3pt]
%\hspace{2mm} group&(0.007)&(0.010)&(0.004)&(0.004)&(0.018)&(0.011)\\
%%Wage&-0.001&-0.002&0.000&0.004&-0.004&0.006\\[-3pt]
%%&(0.001)&(0.003)&(0.001)&(0.002)&(0.006)&(0.006)\\
%\multicolumn{2}{c}{\textbf{F test of interactions$^{a}$}}\\
%Age&1.34&1.02&0.74&1.51&6.99&4.84\\[-3pt]
%\hspace{2mm} [$p$-value]&(0.244)&(0.405)&(0.596)&(0.183)&(0.221)&(0.436)\\
%Age$^{2}$/100&1.10&1.07&0.76&1.52&4.75&4.48\\[-3pt]
%\hspace{2mm} [$p$-value]&(0.361)&(0.376)&(0.582)&(0.181)&(0.447)&(0.483)\\
%Education&1.22&2.04&1.13&0.82&30.01&21.65\\[-3pt]
%\hspace{2mm} [$p$-value]&(0.303)&(0.087)&(0.343)&(0.513)&(0.000)&(0.001)\\
%Non-white&0.78&1.11&2.07&0.66&5.52&6.13\\[-3pt]
%\hspace{2mm} [$p$-value]&(0.539)&(0.349)&(0.083)&(0.619)&(0.356)&(0.294)\\
%\multicolumn{2}{c}{\textbf{Joint significance$^{a}$}}\\
%All IVs&5.58&6.77&3.02&5.30&184.83&171.27\\[-3pt]
%\hspace{2mm} [$p$-value]&(0.000)&(0.000)&(0.000)&(0.000)&(0.000)&(0.000)\\
%Number of&&&&&&\\[-3pt]
%\hspace{2mm}observations&4469&2451&2710&4210&6920&6920\\
%\bottomrule \multicolumn{7}{l}{\footnotesize Note that since there
%are 5 periods
%in MEPS, there are 4 interaction terms for each variable and the}\\[-3pt]
%\multicolumn{7}{l}{\footnotesize  results are joint test of these interactions.}\\[-3pt]
%\end{longtable}
%%\begin{tablenotes}
%%\footnotesize
%%%\noindent
%%Note: $^{a}$Variables that are not changed over time was dropped in the share equation. Only coefficients of excluded instrument variables are reported. Standard errors presented in parentheses are robust to arbitrary heteroskedasticity and correlation within households over time. * significant at $10\%$; ** significant at $5\%$; *** significant at $1\%$.
%%%\end{tablenotes}

%\begin{tablenotes}
%\begin{center}
%\scriptsize \noindent
%Robust standard errors are presented in parentheses. \\
%* significant at $10\%$; ** significant at $5\%$; *** significant at $1\%$.
%\end{center}

%\end{tablenotes}
\normalsize

\newpage

\begin{longtable}{@{}l.{3}@{}.{3}}\label{Table_4}\\
\caption{Pooled Probit IV Regressions: Dependent Variables (EPHI Offer)} \\
\multicolumn{3}{l}{}\\
\toprule
&\multicolumn{1}{r}{Wives}&\multicolumn{1}{r}{Husbands}\\
\hline
Husbands offer&\llap{-}0.496^{*}&\llap{--}\\[-3pt]
&(0.284)&\llap{--}\\
Wife offer&\llap{--}&\llap{-}0.420^{*}\\[-3pt]
&\llap{--}&(0.225)\\
Employed&5.578^{***}&5.801^{***}\\[-3pt]
&(0.200)&(0.270)\\
Tenure&0.002&0.002\\[-3pt]
&(0.005)&(0.003)\\
Wage&0.002&0.014^{*}\\[-3pt]
&(0.007)&(0.007)\\
Pension plan&0.704^{***}&0.857^{***}\\[-3pt]
&(0.204)&(0.204)\\
Sick pay&0.693^{***}&0.557^{***}\\[-3pt]
&(0.209)&(0.160)\\
Kid0-18&\llap{-}0.074&0.043\\[-3pt]
&(0.082)&(0.057)\\
Perceived health &0.029&0.017\\[-3pt]
\hspace{2mm}status&(0.048)&(0.037)\\
Regional unemp.&\llap{-}0.218&0.133\\[-3pt]
\hspace{2mm}rate (\%) &(0.221)&(0.213)\\
Test of fixed effect&64.01&66.56\\[-3pt]
\hspace{2mm} ($p$-value)&(0.002)&(0.001)\\
APE for spouse offer&\llap{-}0.053&\llap{-}0.064\\
Human capital&\multicolumn{1}{c}{Yes}&\multicolumn{1}{c}{Yes}\\
\hspace{2mm}characteristics&&\\
Region dummies&\multicolumn{1}{c}{Yes}&\multicolumn{1}{c}{Yes}\\
\midrule
Number of observations&\multicolumn{1}{c}{6920}&\multicolumn{1}{c}{6920}\\
\bottomrule  \multicolumn{3}{l}{\scriptsize Robust standard errors
are presented in parentheses.}\\[-5pt]
\multicolumn{3}{l}{\scriptsize * significant at 10\%; ** significant
at 5\%; *** signif-}\\[-5pt]
\multicolumn{3}{l}{\scriptsize icant at 1\%.}\\[-5pt]
\multicolumn{3}{l}{\scriptsize Region dummies include Northeast, Midwest, and West.}\\[-3pt]
\multicolumn{3}{l}{\scriptsize Human capital characteristics contain interactions of }\\[-6pt]
\multicolumn{3}{l}{\scriptsize Age, Age$^{2}$/100, Education, and Non-white  with }\\[-6pt]
\multicolumn{3}{l}{\scriptsize time dummies.}\\[-3pt]
\end{longtable}
%%\begin{tablenotes}
%\begin{center}
%\scriptsize \noindent Robust standard errors are presented in
%parentheses. * significant at $10\%$; ** significant at $5\%$; ***
%significant at $1\%$.
%\end{center}

\normalsize


\newpage

\begin{longtable}{@{}l@{}.{3}@{}.{3}@{}.{3}@{}.{3}}\label{Table_5}\\
\caption{Pooled Two Stage Least Square Regressions: Dependent
Variables (Income Share)
}\\
%The effect of spouse's offer on own income share: Dependent Variables (Income Share)
\multicolumn{3}{r}{}\\
\toprule &\multicolumn{2}{r}{Female share
equation}&\multicolumn{2}{r}{Male share
equation}\\[-3pt]
\cmidrule(r){2-3}\cmidrule(r){4-5}
&\multicolumn{1}{r}{Subsample}&\multicolumn{1}{r}{Subsample}&\multicolumn{1}{r}{Subsample}&\multicolumn{1}{r}{Subsample}\\[-3pt]
&\multicolumn{1}{r}{(1)}&\multicolumn{1}{r}{(2)}&\multicolumn{1}{r}{(3)}&\multicolumn{1}{r}{(4)}\\[-3pt]
\hline
Husband's offer&\llap{-}0.123^{***}&\llap{-}0.178^{***}&\llap{--}&\llap{--}\\[-3pt]
&(0.050)&(0.052)&\llap{--}&\llap{--}\\
Wife's offer&\llap{--}&\llap{--}&\llap{-}0.184^{***}&\llap{-}0.427^{***}\\[-3pt]
&\llap{--}&\llap{--}&(0.056)&(0.063)\\
Employed&0.119^{***}&\llap{--}&0.119^{***}&\llap{--}\\[-3pt]
&(0.021)&\llap{--}&(0.029)&\llap{--}\\
Tenure&0.000&0.000&0.000&0.001^{**}\\[-3pt]
&(0.000)&(0.001)&(0.001)&(0.001)\\
Wage&0.000&0.000&\llap{-}0.001&0.002\\[-3pt]
&(0.001)&(0.001)&(0.001)&(0.002)\\
Pension plan&0.021&\llap{-}0.004&\llap{-}0.099^{***}&\llap{-}0.011\\[-3pt]
&(0.031)&(0.045)&(0.031)&(0.035)\\
Sick pay&\llap{-}0.007&0.008&0.066^{**}&0.004\\[-3pt]
&(0.040)&(0.054)&(0.029)&(0.040)\\
Kid0-18&0.000&\llap{-}0.029^{*}&\llap{-}0.001&\llap{-}0.003\\[-3pt]
&(0.000)&(0.018)&(0.018)&(0.015)\\
Perceived health&0.000&0.001&\llap{-}0.019^{**}&0.017^{**}\\[-3pt]
\hspace{2mm} status&(0.006)&(0.008)&(0.009)&(0.009)\\
Regional unemp.&0.016&0.001&\llap{-}0.030&\llap{-}0.049\\[-3pt]
\hspace{2mm}rate (\%)&(0.047)&(0.053)&(0.077)&(0.052)\\
Human capital&\multicolumn{1}{r}{Yes}&\multicolumn{1}{r}{Yes}&\multicolumn{1}{r}{Yes}&\multicolumn{1}{r}{Yes}\\[-4pt]
\hspace{2mm}characteristics&&\\
Region dummies&\multicolumn{1}{r}{Yes}&\multicolumn{1}{r}{Yes}&\multicolumn{1}{r}{Yes}&\multicolumn{1}{r}{Yes}\\
Test of overidentifying&13.312&17.517&14.657&13.883\\[-3pt]
\hspace{2mm}restriction [$p$-value]&0.822&0.555&0.744&0.791\\
\midrule
Number of households&\multicolumn{1}{r}{4420} &\multicolumn{1}{r}{2221}&\multicolumn{1}{r}{2630}&\multicolumn{1}{r}{4043} \\
\bottomrule \multicolumn{5}{l}{\scriptsize Note: robust standard
errors are presented in parentheses.
* significant at 10\%; ** significant}\\[-6pt]
\multicolumn{5}{l}{\scriptsize  at 5\%; *** significant at 1\%.
}\\[-3pt]
\multicolumn{5}{l}{\scriptsize Subsample (1), subsample (2),
subsample (3), and subsample (4)
 denote the following four}\\[-6pt]
\multicolumn{5}{l}{\scriptsize  groups, female without EHPI offer,
female with EHPI offer, male without EHPI offer and }\\[-6pt]
\multicolumn{5}{l}{\scriptsize  male with EHPI offer, respectively.}\\[-3pt]
\multicolumn{5}{l}{\scriptsize Region dummies include Northeast, Midwest, South, and West.}\\[-3pt]
\multicolumn{5}{l}{\scriptsize Human capital characteristics contain interactions of Age, Age$^{2}$/100, Education, and Non-white }\\[-6pt]
\multicolumn{5}{l}{\scriptsize with time dummies.}\\[-3pt]
\end{longtable}

\newpage
\begin{figure}
\begin{center}
\includegraphics[width=0.7\textwidth]{figure1.pdf}

\caption{The Cooperative Household EPHI Access Decisions}\label{figure1}
\end{center}
\end{figure}



\end{document}
